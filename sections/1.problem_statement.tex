\section{Research Goals}\label{sec:research_goals}
Mental disorders (e.g. schizophrenia) are complex phenotypes\cite{insel2009disruptive,kapur2012has,stephan2015translational,hitchcock2022computational}, yet their diagnosis and treatment are solely based on the assessment of symptoms\cite{kapur2012has,stephan2015translational,hitchcock2022computational,scarr2015biomarkers}. The dynamic nature of symptom development, coupled with the lack of objective diagnostic tests, can lead to misdiagnosis, particularly in children and minority groups\cite{merten2017overdiagnosis,gara2019naturalistic}. Consequentially, growing interest in developing reliable quantitative biomarkers for detecting related symptoms has recently emerged. Biomarker studies rely on the dominant case-control paradigm, for which a group of healthy individuals is compared to a group of patients, which generally occludes individual-level differences contributing the development of a complex brain disorders. These disorders do not follow a clear-cut distinction between cases-controls but are more complex in terms of biology and aetiology. Therefore, we develop a set of machine learning methods, inspired by approaches used in paediatrics for which a child’s height is charted against age. In our team we have earlier developed methods under the banner of normative modelling, for which we mapped individual differences in reference to a population for more than a hundred thousand different brain imaging-based biomarkers  6–9. We could show that only about 2-10\% of the individuals with schizophrenia show extreme deviations in the same brain regions (REF). This finding generalizes to other complex brain disorders and diseases and has been replicated extensively 6,9–11. We recently showed that quantifying deviations in brain structure at the level of the brain network led to an increase of overlap between patients 12, highlighting the possibility that mental disorders are network disorders 13. Here we propose innovation based on ...


Challenges in current researches
Data-wise:

1. case-control paradigm: The dominant case-control paradigm, for which a group of healthy individuals is compared to a group of patients, occludes individual-level differences contributing to disorder or disease.
Method-wise: supervised machine learning

1. Requires large matched patient and control datasets, which can be challenging and costly to obtain

2. does not account Heterogeneity / comorbidity across different mental disorders

Proposed:
Requires exploring a robust and reliable computational method to model individual Heterogeneity across biological factors
-> aiding in the identification of novel illness subtypes and enhancing generalizability to new data and populations without relying on labeled supervision.

