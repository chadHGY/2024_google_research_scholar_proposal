% \section{Research Statement}\label{sec:research_statement}
\section{Research Statement}\label{sec:research_statement}
Mental disorders, such as schizophrenia, pose complex challenges in diagnosis and treatment due to their multifaceted nature and reliance on symptom assessments \cite{kapur2012has,scarr2015biomarkers,stephan2015translational,hitchcock2022computational}. This reliance often leads to misdiagnosis, especially among children and minority groups, highlighting the critical need for objective biomarkers in mental health diagnostics\cite{merten2017overdiagnosis,gara2019naturalistic}. Current biomarker research faces limitations stemming from conventional case-control paradigms and supervised modeling techniques. The dominant case-control approach, limited by biased data selection and challenges in establishing causation, restricts the generalizability of findings. Similarly, supervised modeling techniques encounter scarcity or bias in labeled data, potentially leading to overfitting, ethical concerns, and oversimplified representations of mental health complexities. Addressing these limitations requires a comprehensive approach integrating diverse methodologies and high-quality, diverse datasets to ensure accuracy, generalizability, and ethical considerations in predictive models for mental health research. This endeavor not only aims to develop objective biomarkers but also seeks to impact broader mental health populations, emphasizing the collaborative effort needed to advance the field.



% Challenges in current researches
% Data-wise:

% 1. case-control paradigm: The dominant case-control paradigm, for which a group of healthy individuals is compared to a group of patients, occludes individual-level differences contributing to disorder or disease.
% Method-wise: supervised machine learning

% 1. Requires large matched patient and control datasets, which can be challenging and costly to obtain

% 2. does not account Heterogeneity / comorbidity across different mental disorders

% Proposed:
% Requires exploring a robust and reliable computational method to model individual Heterogeneity across biological factors
% -> aiding in the identification of novel illness subtypes and enhancing generalizability to new data and populations without relying on labeled supervision.

